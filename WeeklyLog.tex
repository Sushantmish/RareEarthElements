% Options for packages loaded elsewhere
\PassOptionsToPackage{unicode}{hyperref}
\PassOptionsToPackage{hyphens}{url}
%
\documentclass[
]{article}
\usepackage{amsmath,amssymb}
\usepackage{iftex}
\ifPDFTeX
  \usepackage[T1]{fontenc}
  \usepackage[utf8]{inputenc}
  \usepackage{textcomp} % provide euro and other symbols
\else % if luatex or xetex
  \usepackage{unicode-math} % this also loads fontspec
  \defaultfontfeatures{Scale=MatchLowercase}
  \defaultfontfeatures[\rmfamily]{Ligatures=TeX,Scale=1}
\fi
\usepackage{lmodern}
\ifPDFTeX\else
  % xetex/luatex font selection
\fi
% Use upquote if available, for straight quotes in verbatim environments
\IfFileExists{upquote.sty}{\usepackage{upquote}}{}
\IfFileExists{microtype.sty}{% use microtype if available
  \usepackage[]{microtype}
  \UseMicrotypeSet[protrusion]{basicmath} % disable protrusion for tt fonts
}{}
\makeatletter
\@ifundefined{KOMAClassName}{% if non-KOMA class
  \IfFileExists{parskip.sty}{%
    \usepackage{parskip}
  }{% else
    \setlength{\parindent}{0pt}
    \setlength{\parskip}{6pt plus 2pt minus 1pt}}
}{% if KOMA class
  \KOMAoptions{parskip=half}}
\makeatother
\usepackage{xcolor}
\usepackage[margin=1in]{geometry}
\usepackage{color}
\usepackage{fancyvrb}
\newcommand{\VerbBar}{|}
\newcommand{\VERB}{\Verb[commandchars=\\\{\}]}
\DefineVerbatimEnvironment{Highlighting}{Verbatim}{commandchars=\\\{\}}
% Add ',fontsize=\small' for more characters per line
\usepackage{framed}
\definecolor{shadecolor}{RGB}{248,248,248}
\newenvironment{Shaded}{\begin{snugshade}}{\end{snugshade}}
\newcommand{\AlertTok}[1]{\textcolor[rgb]{0.94,0.16,0.16}{#1}}
\newcommand{\AnnotationTok}[1]{\textcolor[rgb]{0.56,0.35,0.01}{\textbf{\textit{#1}}}}
\newcommand{\AttributeTok}[1]{\textcolor[rgb]{0.13,0.29,0.53}{#1}}
\newcommand{\BaseNTok}[1]{\textcolor[rgb]{0.00,0.00,0.81}{#1}}
\newcommand{\BuiltInTok}[1]{#1}
\newcommand{\CharTok}[1]{\textcolor[rgb]{0.31,0.60,0.02}{#1}}
\newcommand{\CommentTok}[1]{\textcolor[rgb]{0.56,0.35,0.01}{\textit{#1}}}
\newcommand{\CommentVarTok}[1]{\textcolor[rgb]{0.56,0.35,0.01}{\textbf{\textit{#1}}}}
\newcommand{\ConstantTok}[1]{\textcolor[rgb]{0.56,0.35,0.01}{#1}}
\newcommand{\ControlFlowTok}[1]{\textcolor[rgb]{0.13,0.29,0.53}{\textbf{#1}}}
\newcommand{\DataTypeTok}[1]{\textcolor[rgb]{0.13,0.29,0.53}{#1}}
\newcommand{\DecValTok}[1]{\textcolor[rgb]{0.00,0.00,0.81}{#1}}
\newcommand{\DocumentationTok}[1]{\textcolor[rgb]{0.56,0.35,0.01}{\textbf{\textit{#1}}}}
\newcommand{\ErrorTok}[1]{\textcolor[rgb]{0.64,0.00,0.00}{\textbf{#1}}}
\newcommand{\ExtensionTok}[1]{#1}
\newcommand{\FloatTok}[1]{\textcolor[rgb]{0.00,0.00,0.81}{#1}}
\newcommand{\FunctionTok}[1]{\textcolor[rgb]{0.13,0.29,0.53}{\textbf{#1}}}
\newcommand{\ImportTok}[1]{#1}
\newcommand{\InformationTok}[1]{\textcolor[rgb]{0.56,0.35,0.01}{\textbf{\textit{#1}}}}
\newcommand{\KeywordTok}[1]{\textcolor[rgb]{0.13,0.29,0.53}{\textbf{#1}}}
\newcommand{\NormalTok}[1]{#1}
\newcommand{\OperatorTok}[1]{\textcolor[rgb]{0.81,0.36,0.00}{\textbf{#1}}}
\newcommand{\OtherTok}[1]{\textcolor[rgb]{0.56,0.35,0.01}{#1}}
\newcommand{\PreprocessorTok}[1]{\textcolor[rgb]{0.56,0.35,0.01}{\textit{#1}}}
\newcommand{\RegionMarkerTok}[1]{#1}
\newcommand{\SpecialCharTok}[1]{\textcolor[rgb]{0.81,0.36,0.00}{\textbf{#1}}}
\newcommand{\SpecialStringTok}[1]{\textcolor[rgb]{0.31,0.60,0.02}{#1}}
\newcommand{\StringTok}[1]{\textcolor[rgb]{0.31,0.60,0.02}{#1}}
\newcommand{\VariableTok}[1]{\textcolor[rgb]{0.00,0.00,0.00}{#1}}
\newcommand{\VerbatimStringTok}[1]{\textcolor[rgb]{0.31,0.60,0.02}{#1}}
\newcommand{\WarningTok}[1]{\textcolor[rgb]{0.56,0.35,0.01}{\textbf{\textit{#1}}}}
\usepackage{graphicx}
\makeatletter
\newsavebox\pandoc@box
\newcommand*\pandocbounded[1]{% scales image to fit in text height/width
  \sbox\pandoc@box{#1}%
  \Gscale@div\@tempa{\textheight}{\dimexpr\ht\pandoc@box+\dp\pandoc@box\relax}%
  \Gscale@div\@tempb{\linewidth}{\wd\pandoc@box}%
  \ifdim\@tempb\p@<\@tempa\p@\let\@tempa\@tempb\fi% select the smaller of both
  \ifdim\@tempa\p@<\p@\scalebox{\@tempa}{\usebox\pandoc@box}%
  \else\usebox{\pandoc@box}%
  \fi%
}
% Set default figure placement to htbp
\def\fps@figure{htbp}
\makeatother
\setlength{\emergencystretch}{3em} % prevent overfull lines
\providecommand{\tightlist}{%
  \setlength{\itemsep}{0pt}\setlength{\parskip}{0pt}}
\setcounter{secnumdepth}{-\maxdimen} % remove section numbering
\usepackage{bookmark}
\IfFileExists{xurl.sty}{\usepackage{xurl}}{} % add URL line breaks if available
\urlstyle{same}
\hypersetup{
  pdftitle={Weekly Log},
  pdfauthor={Sushant},
  hidelinks,
  pdfcreator={LaTeX via pandoc}}

\title{Weekly Log}
\author{Sushant}
\date{2025-10-02}

\begin{document}
\maketitle

\textbf{Week 1 Reading Notes:}\\
\textbf{\emph{Book: Using Geochemical Data (Rollinson and Pease)}}\\
1. Normalising values can be chosen from 2 sets:\\
i) CI chondrites\\
ii) Ordinary chrondrites (volatile free, tend to have higher values).\\
Therefore, it is important to cite the right data set. See pg.130 for
dataset.

Shale normalisation is used for upper crust sediments while Chondrite
normalisation will be used in our project since we want to compare
igneous rocks to primitive solar system.

\begin{enumerate}
\def\labelenumi{\arabic{enumi}.}
\setcounter{enumi}{1}
\item
  REE in minerals in Igneous Rocks:\\
  i)\textbf{Feldspars}: low concentration, but when in felsic rocks, Eu
  anomaly is +ve\\
  ii)\textbf{Olivine}: very low concentration, unlikely to fractionate
  REE while fractionation/partial melting.\\
  iii)\textbf{Pyroxenes}: orthopyroxenes have lower REE than
  clinopyroxenes. In mafic rocks, REE are mostly present in
  clinopyroxene. Pyroxenes have small -ve anomalies but maybe be higher
  in clinopyroxenes.\\
  iv)\textbf{Hornblende}: enriched in middle REE, have higher
  concentration of HREE that forms a U-shaped REE pattern in melt.\\
  v)\textbf{Garnet}: high abundance of HREE to the extent that they are
  compatible whilst LREE aren't.\\
  vi)\textbf{Muscovite}: enriched in REE, especially in felsic rocks.
  -ve Eu anomaly.\\
  vii)\textbf{Biotite}: less enriched in REE, -ve Eu anomaly.\\
  viii)\textbf{Zircon}: deplete a melt in HREE.\\
  ix)\textbf{Other accessory phases}: strong influence on REE pattern
  even if they have low abundance. They have high partition coefficients
  which result in disproportionate influence and depletion in REE.
\item
  Multi-element Diagrams (MeD):\\
  i)\textbf{Chondrite-Normalised MeD}: advantageous since it uses
  directly measured values instead of calculated ones. It might not be
  very suitable for sediment and evolved igneous rocks ( do they mean
  metamorphic?). They have plotted REE and and some other trace elements
  in order of increasing mantle compatibility on log scale using
  normalised values with McDonough and and Sun(1995) and Palme and
  ONeil(2014) as reference values.\\
  ii)In evolved igenous rocks, anomalies may be controlled by specific
  mineral as mentioned in point 2 above where LREE are controlled by
  allanite and HREE by garnet.
\end{enumerate}

\textbf{\emph{Research Paper- Wall (2012) :``Rare Earth Elements:
Minerals, Mines, Magnets''}}

\textbf{\emph{Research Paper- O'Neill(2016) :``The Smoothness and Shapes
of Chondrite-normalized Rare Earth Element Patterns in Basalts''}}

\textbf{\emph{Code for Week 1: Spider Plot Function}}

In a Spider Diagram, we will be representing the log normalised
concentrations of Rare Earth Elements. The following Elements will be
plotted: La-Lanthanum, Ce- Cerium, Pr- Praseodymium, Nd- Neodymium, Pm-
Promethium, Sm- Samarium, Eu- Europium, Gd- Gadolinium,, Tb- Terbium,
Dy- Dysprosium, Ho- Holmium, Er- Erbium, Tm- Thulium, Yb- Ytterbium, Lu-
Lutetium

\begin{Shaded}
\begin{Highlighting}[]
\CommentTok{\#downloading dataset}
\NormalTok{adakite }\OtherTok{\textless{}{-}} \FunctionTok{read.csv}\NormalTok{(}\AttributeTok{file =} \StringTok{\textquotesingle{}Data/ADAKITE.csv\textquotesingle{}}\NormalTok{)}
\NormalTok{ref }\OtherTok{\textless{}{-}} \FunctionTok{read.csv}\NormalTok{(}\AttributeTok{file=}\StringTok{\textquotesingle{}Data/Normalised Values.csv\textquotesingle{}}\NormalTok{)}
\CommentTok{\# gathering REE data from sample}
\NormalTok{ree\_sample1 }\OtherTok{\textless{}{-}} \FunctionTok{as.numeric}\NormalTok{(adakite[}\DecValTok{19}\NormalTok{, }\DecValTok{119}\SpecialCharTok{:}\DecValTok{132}\NormalTok{])}
\CommentTok{\#}
\NormalTok{ree\_ref }\OtherTok{\textless{}{-}} \FunctionTok{as.numeric}\NormalTok{(ref[}\DecValTok{3}\SpecialCharTok{:}\DecValTok{16}\NormalTok{, }\DecValTok{14}\NormalTok{]) }\CommentTok{\#column 4,5 dont work}

\CommentTok{\#this is a function for plotting spider plots for REE data}
\NormalTok{plot\_spider }\OtherTok{\textless{}{-}} \ControlFlowTok{function}\NormalTok{(samples,reference\_values,}\AttributeTok{logscale=}\ConstantTok{TRUE}\NormalTok{,}\AttributeTok{colors=}\ConstantTok{NULL}\NormalTok{,}\AttributeTok{main=} \StringTok{"Spider Plot for Normalised REE Values"}\NormalTok{,}\AttributeTok{xlab=}\StringTok{"Rare Earth Elements"}\NormalTok{,}\AttributeTok{ylab =}\StringTok{"Normalised values"}\NormalTok{)}
\NormalTok{\{}
  \CommentTok{\# these are the REE we will be covering generally}
\NormalTok{  REE\_symbols }\OtherTok{\textless{}{-}}  \FunctionTok{c}\NormalTok{(}\StringTok{"La"}\NormalTok{,}\StringTok{"Ce"}\NormalTok{,}\StringTok{"Pr"}\NormalTok{,}\StringTok{"Nd"}\NormalTok{,}\StringTok{"Sm"}\NormalTok{,}\StringTok{"Eu"}\NormalTok{,}\StringTok{"Gd"}\NormalTok{,}\StringTok{"Tb"}\NormalTok{,}\StringTok{"Dy"}\NormalTok{,}\StringTok{"Ho"}\NormalTok{,}\StringTok{"Er"}\NormalTok{,}\StringTok{"Tm"}\NormalTok{,}\StringTok{"Yb"}\NormalTok{,}\StringTok{"Lu"}\NormalTok{)}
  \CommentTok{\# normalising values of samples with the reference set}
\NormalTok{  normalised\_values }\OtherTok{\textless{}{-}} \FunctionTok{lapply}\NormalTok{(samples,}\ControlFlowTok{function}\NormalTok{(s1) }\FunctionTok{as.numeric}\NormalTok{(s1)}\SpecialCharTok{/}\FunctionTok{as.numeric}\NormalTok{(reference\_values))}
  \CommentTok{\#axis range}
\NormalTok{  ymin }\OtherTok{\textless{}{-}} \ControlFlowTok{if}\NormalTok{ (logscale) }\FunctionTok{min}\NormalTok{(}\FunctionTok{unlist}\NormalTok{(normalised\_values), }\AttributeTok{na.rm =} \ConstantTok{TRUE}\NormalTok{) }\ControlFlowTok{else} \DecValTok{0}
\NormalTok{  ymax }\OtherTok{\textless{}{-}} \FunctionTok{max}\NormalTok{(}\FunctionTok{unlist}\NormalTok{(normalised\_values), }\AttributeTok{na.rm =} \ConstantTok{TRUE}\NormalTok{)}
  \CommentTok{\# generate plot }
  \FunctionTok{plot}\NormalTok{(}\DecValTok{1}\SpecialCharTok{:}\FunctionTok{length}\NormalTok{(reference\_values),normalised\_values[[}\DecValTok{1}\NormalTok{]],}\AttributeTok{type=}\StringTok{"b"}\NormalTok{,}\AttributeTok{xaxt=}\StringTok{"n"}\NormalTok{,,}\AttributeTok{xlab=}\StringTok{"Rare Earth Elements"}\NormalTok{,}\AttributeTok{ylab =}\StringTok{"Concentration"}\NormalTok{,}\AttributeTok{log =} \StringTok{\textquotesingle{}y\textquotesingle{}}\NormalTok{,}\AttributeTok{ylim =} \FunctionTok{c}\NormalTok{(ymin, ymax))}
  \FunctionTok{axis}\NormalTok{(}\DecValTok{1}\NormalTok{,}\AttributeTok{at=}\DecValTok{1}\SpecialCharTok{:}\FunctionTok{length}\NormalTok{(reference\_values),}\AttributeTok{labels =}\NormalTok{ REE\_symbols)}
  \FunctionTok{grid}\NormalTok{(}\AttributeTok{nx=}\ConstantTok{NULL}\NormalTok{, }\AttributeTok{ny=}\ConstantTok{NULL}\NormalTok{,}\AttributeTok{col=}\StringTok{"gray"}\NormalTok{,}\AttributeTok{lwd=}\DecValTok{1}\NormalTok{)}
  \FunctionTok{abline}\NormalTok{(}\AttributeTok{h=}\DecValTok{1}\NormalTok{,}\AttributeTok{col=}\StringTok{"black"}\NormalTok{,}\AttributeTok{lwd=}\DecValTok{3}\NormalTok{)}
  
  \CommentTok{\# generate other plots }
   \ControlFlowTok{if}\NormalTok{(}\FunctionTok{length}\NormalTok{(normalised\_values)}\SpecialCharTok{\textgreater{}}\DecValTok{1}\NormalTok{)}
\NormalTok{   \{ }
       \ControlFlowTok{for}\NormalTok{(i }\ControlFlowTok{in} \DecValTok{2}\SpecialCharTok{:}\FunctionTok{length}\NormalTok{(normalised\_values))}
\NormalTok{       \{}
         \FunctionTok{lines}\NormalTok{(}\DecValTok{1}\SpecialCharTok{:}\FunctionTok{length}\NormalTok{(normalised\_values),normalised\_values[[i]],}\AttributeTok{type=}\StringTok{"b"}\NormalTok{)}
\NormalTok{       \}}
\NormalTok{   \}}
\NormalTok{\}}
\NormalTok{samples }\OtherTok{\textless{}{-}} \FunctionTok{list}\NormalTok{(ree\_sample1)}

\FunctionTok{plot\_spider}\NormalTok{(samples, ree\_ref,}\AttributeTok{main =} \StringTok{"REE Spider Diagram (Sample 19, Reference: m\&s )"}\NormalTok{)}
\end{Highlighting}
\end{Shaded}

\pandocbounded{\includegraphics[keepaspectratio]{WeeklyLog_files/figure-latex/unnamed-chunk-1-1.pdf}}

\end{document}
